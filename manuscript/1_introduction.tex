\section{Introduction}
Diabetes is regarded as a global eye health issue, with a steadily increasing world-wide affected population, estimated to reach 630 million individuals by 2045 \cite{noauthor_who_nodate}. Diabetic Retinopathy (DR) is a complication of standard diabetes, caused by damage to vasculature within the retina. DR shows early signs in the form of swelling micro-lesions that destroy small vessels and release blood into the retina. More advanced DR stages are characterized by the appearance of other more noticeable symptoms, e.g. proliferation of neo-vessels, which may lead to the detachment of the retinal layer and eventually
permanent sight loss. Retinal images acquired with fundus cameras are the tool of choice for capturing and depicting all these early symptoms, thereby representing an effective diagnostic tool \cite{fenner_advances_2018}. 

This challenge is composed by three sub-challenges, namely DR grading from standard fundus images (Challenge 1), from Ultra-Wield Field retinal images (Challenge 3), and retinal fundus image quality assessment (Challenge 2). In addition, for each eye the organization provides a retinal image acquired when the optic disc is located in the center of the picture, and a complementary image where the acquisition is with the macula on the center, as shown in Figure 1. Our team has only participated in the first Challenge, although we expect to participate on the other two challenges in the next days. For this reason, we only include a description of our approach for Challenge 1, but we will expand this document to account for the other tasks upon our further submissions.