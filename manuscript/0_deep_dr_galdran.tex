\documentclass{article}
\usepackage{spconf,amsmath,graphicx}
\usepackage[utf8]{inputenc}
\usepackage{xcolor}
\usepackage{blindtext}
\usepackage{enumitem}
\usepackage{booktabs}
\usepackage{hyperref}
\usepackage[caption=false]{subfig}
\usepackage{caption} 
\captionsetup[table]{skip=8pt}
%\usepackage{makecell}
%\renewcommand\theadalign{bc}
%\renewcommand\theadfont{\bfseries}
%\renewcommand\theadgape{\Gape[4pt]}
%\renewcommand\cellgape{\Gape[4pt]}
% For subfig.sty:
 \let\MYorigsubfloat\subfloat
 \renewcommand{\subfloat}[2][\relax]{\MYorigsubfloat[]{#2}}
% Example definitions.
% --------------------
\def\x{{\mathbf x}}
\def\L{{\cal L}}
\usepackage{dblfloatfix}

\newcommand{\ines}{\textcolor{black}}

% Title.
% ------
\title{COST-SENSITIVE LOSS MINIMIZATION FOR RETINAL IMAGE ANALYSIS}


\name{Adrian Galdran$^{\star, \dagger}$, J. Dolz$^{\dagger}$, A. Christodoulidis$^{\ddagger}$, H. Chakor$^{\ddagger}$, H. Lombaert$^{\dagger}$, I. ben Ayed$^{\dagger}$
\thanks{Corresponding author: Adrian Galdran (agaldran@bournemouth.ac.uk).
}
}
\address{$^{\star}$ 
Bournemouth University, UK \\
$^{\dagger}$ École de Technologie Supérieure, University of Quebec, Canada
\\
${\ddagger}$ Diagnos INC., Quebec, Canada
}


\begin{document}
%\ninept
%
\maketitle
%
\begin{abstract}
This is a short description of the method employed by our team to address the DEEP-DR challenge. It will be expanded and improved in the following days to account for our full solution including the three sub-challenges.
\end{abstract}
%
\begin{keywords}
Diabetic Retinopathy Grading, Cost-Sensitive Classification, Retinal Image Quality Assessment, Ultra-Wide Field Retinal Imaging
\end{keywords}

\begin{figure*}[t]
\centering
\subfloat[]{\includegraphics[width = 0.24\textwidth]{images/12_r1.jpg}
\label{fig_deg_1}}
\hfil
\subfloat[]{\includegraphics[width = 0.24\textwidth]{images/12_r2.jpg}
\label{fig_deg_2}}
\hfil
\subfloat[]{\includegraphics[width = 0.24\textwidth]{images/18_l2.jpg}
\label{fig_deg_3}}
\hfil
\subfloat[]{\includegraphics[width = 0.24\textwidth]{images/18_r1.jpg}
\label{fig_deg_4}}
\caption{Different image modalities involved in this challenge: (a) Optic Disc-centered retinal fundus image, (b) Macula-centered retinal fundus image, (c) Optic Disc-centered UW-Field retinal image, and (d) Macula-centered Ultra-Wild Field retinal image.}
\label{fig_samples}
\end{figure*}

\section{Introduction}
Diabetes is regarded as a global eye health issue, with a steadily increasing world-wide affected population, estimated to reach 630 million individuals by 2045 \cite{noauthor_who_nodate}. Diabetic Retinopathy (DR) is a complication of standard diabetes, caused by damage to vasculature within the retina. DR shows early signs in the form of swelling micro-lesions that destroy small vessels and release blood into the retina. More advanced DR stages are characterized by the appearance of other more noticeable symptoms, e.g. proliferation of neo-vessels, which may lead to the detachment of the retinal layer and eventually
permanent sight loss. Retinal images acquired with fundus cameras are the tool of choice for capturing and depicting all these early symptoms, thereby representing an effective diagnostic tool \cite{fenner_advances_2018}. 

This challenge is composed by three sub-challenges, namely DR grading from standard fundus images (Challenge 1), from Ultra-Wield Field retinal images (Challenge 3), and retinal fundus image quality assessment (Challenge 2). 
In addition, for each eye the organization provides a retinal image acquired when the optic disc is located in the center of the picture, and a complementary image where the acquisition is with the macula on the center, as shown in Figure 1. Our team  participates in all three Challenges. 
In the following we detail our proposed solution for each task.

\section{Methodology}
In this work, we consider a standard CNN archtitecture, namely ResNet-50 \cite{he_deep_2016}, as it
has established in recent years as a standard in computer vision applications. \textit{ResNet 50}, introduced in \cite{he_deep_2016}, is part of the family of residual neural networks, the key contribution of which was the addition of \textit{skip connections}. Skip connections address the well-known vanishing gradient problem: in very deep neural networks, the weights in early layers of the network are not properly updated. This is due to the backpropagation algorithm propagating increasingly smaller error gradient values. Skip connections help in preserving the error gradient by allowing to backpropagate through an identity mapping instead of through standard layers. 
This is achieved within a residual block through the mapping of input features $x$ to output features $H(x)$ by means of the following formula:
\begin{equation*}
H(x) = F(x)+x,
\end{equation*}
where $x\mapsto F(x)$ is a standard neural network layer. In this case, if during the training stage it is found that backpropagating the error signal through $F$ is harmful for the model performance, the training process can be automatically corrected to deviate through the identity mapping $H(x)=x$, which will not modify the error gradient values at all.

By stacking residual blocks, ResNets with up to several hundred convolutional layers can be trained. 
However, no significant improvement is achieved by using such a large amount of layers. 
Hence, in this work we restrict ourselves to a 50-layers residual network, which is one of the standard architectures employed nowadays in computer vision.

Our main novelty (submitted for consideration to MICCAI 2020) is the introduction of Cost-Sensitive Regularization in the training process. We consider as our base loss function $\mathcal{L}$ the well-known focal loss [X]:
\begin{equation}
\mathcal{L}(p,y) = -\alpha (y \cdot p)^\gamma \log(p)
\end{equation}
We then expand the above loss function by a cost-sensitive term given by:
\begin{equation}
\mathcal{L}_{CS}(p,y) = \mathcal{L}(p,y) + \lambda \langle M[y[i],:]\cdot x\rangle
\end{equation}
where $M$ is a ground-cost matrix defined as:
\begin{equation}
M(i,j) = \|i-j\|,
\end{equation}
and $M[y[i],:]$ denotes the row in $M$ associated to the corresponding label.

Our model was trained based on the above loss function and architecture for each of the two image subsets (OD-centered and macula-centered), and predictions on each subset where combined into our submission. Training details and extra technical clarifications will be provided in future versions of this document.

%\subsection{Fine-Tuning for UW-image recognition}
%All three CNN models used in this work were pre-trained on ImageNet and fine-tuned to the available training data by minimizing the cross entropy loss.
%We used the Adam \cite{kingma_adam:_2015} optimizer with a learning rate of $1E^{-3}$ and default $\beta_1$ and $\beta_2$ values ($\beta_1 = 0.9$, $\beta_2 = 0.999$).
%The learning rate was decayed by $0.1$ after every $7$ epochs.
%Early stopping was used with a patience of $5$ by monitoring Area Under the ROC Curve (AUC) on a separate validation set.
%Input images were resized to $299\times487$ pixels to keep the original aspect ratio.
%Finally, we used standard dataset augmentation operations such as random translations, scaling and horizontal flips.




%
%\begin{table*}[!b]  %* for the two columns
	%% increase table row spacing, adjust to taste
	%\renewcommand{\arraystretch}{1.3}	
	\centering
	%\sisetup{detect-weight=true,detect-inline-weight=math}
%\setlength\tabcolsep{5pt}	
\caption{Performance comparison between the technique from \cite{moccia_learning-based_2018} and fine-tuned SqueezeNet. \textbf{IQR}: Inter-Quartile Range.}
\begin{tabular}{l cccccc c cccccc }
%\toprule
 & \multicolumn{6}{c}{\textbf{Feature-Based + SVM} \cite{moccia_learning-based_2018} } & & \multicolumn{6}{c}{\textbf{Proposed (SqueezeNet-based)}} \\
\cmidrule(lr){2-7} \cmidrule(lr){9-14}
& \textbf{I}   & \textbf{B}  & \textbf{S} & \textbf{U} & \textbf{Median} &\textbf{IQR}& &\textbf{I}   & \textbf{B}  & \textbf{S} & \textbf{U} & \textbf{Median} &\textbf{IQR}\\
%\midrule
\cmidrule(lr){1-7} \cmidrule(lr){9-14}
\textbf{Precision}  & 0.91 & 0.76 & 0.78 & 0.76 & 0.77 & 0.09 & & 0.97 & 0.94 & 0.93 & 0.97 & 0.95 & 0.03 \\
\textbf{Recall}       & 0.91 & 0.83 & 0.62 & 0.85 & 0.84 & 0.16 & & 1 & 0.94 & 0.91 & 0.94 & 0.94 & 0.02 \\
\textbf{F1-Score}   & 0.91 & 0.79 & 0.69 & 0.80 & 0.80 & 0.12 & & 0.98 & 0.94 & 0.91 & 0.95 & 0.95 & 0.03 \\
\bottomrule
\end{tabular}
\label{tab_1_results}
\end{table*}%* for the two columns

%
%\section{Experimental Evaluation}
\subsection{Data}
For benchmarking the three models described in the previous section, we employ the recent introduced NBI-InfFrames dataset \cite{moccia_learning-based_2018}.  This dataset contains $720$ video frames obtained from 18 NBI laryngoscopic videos of 18 different patients. All of them were affected by laryngeal spinocellular carcinoma, as confirmed by further histopathological examination. Video acquisition was performed with a Narrow Band Imaging endoscope at a frame rate of 25 frames per second and a resolution of $1920\times1072$ pixels. From the $720 $ available samples, $180$ were Informative (\textbf{I}), $180$ were considered as Blurred (\textbf{B}), $180$ were declared as containing Saliva or Specular reflections (\textbf{S}), and $180$ were deemed as underexposed (\textbf{U}) by two different medical experts, see Fig. \ref{fig_samples}.

The  NBI-InfFrames dataset comes already divided into three different folds, carefully constructed to separate frames patient-wise into different folds. In our case, for each considered model we performed three different training stages. In each stage, one fold was employed for training the model, another one for validation purposes, and the third fold was used for testing the performance of the model. This was repeated three times, suitably varying the corresponding test fold.



\subsection{Quantitative Evaluation}
For a numerical evaluation of the performance achieved by each model, we computed True Positive Rate and False Positive Rate averaged over the three experiments described above. From this, and given that the dataset was balanced, we built macro-average ROC curves for each of them. The resulting curves are shown in Fig. \ref{fig_rocs}, together with the corresponding Area Under the Curve (AUC) values. 

\begin{figure}[t]
\centering
\includegraphics[width=0.45\textwidth]{images/composed_roc.pdf}
\caption{ROC curves for each of the three considered models.}\label{fig_rocs}
\end{figure}



From the previous experiment, we can observe that the three models achieved a high performance in the task of discriminating among the four classes of interest. Interestingly, the fine-tuned SqueezeNet model obtained the largest performance, with a nearly perfect AUC. For this reason, we selected this model as the best-performing one and further studied its performance in each class of interest, with a similar analysis as the one offered in \cite{moccia_learning-based_2018}. For this, after thresholding predictions at $t=0.5$, we computed True Positives ($\textrm{TP}_j$), False Positives ($\textrm{FP}_j$), and False Negatives ($\textrm{FN}_j$) for each class $j\in\{1,2,3,4\}$. With this, per-class Precision, Recall, and F1 scores were computed as follows:
\begin{gather*} 
\textrm{Precision}_j = \frac{\displaystyle \textrm{TP}_j}{\textrm{TP}_j+\textrm{FP}_j} ,
\quad \quad \quad 
\textrm{Recall}_j = \frac{\displaystyle \textrm{TP}_j}{\displaystyle \textrm{TP}_j+\textrm{FN}_j},
\\
\textrm{F1}_j = 2\cdot
\frac{\displaystyle \textrm{Precision}_j\cdot \textrm{Recall}_j}
{\displaystyle\textrm{Precision}_j + \textrm{Recall}_j}.
\end{gather*}
Table \ref{tab_1_results} shows the result of computing the above performance measures for the SqueezeNet-based model, with results reported in \cite{moccia_learning-based_2018} for the same task and equal experimental setting.



In addition to analyzing how accurate predictions were for each class, we were also interested in studying the time required for each model to produce a prediction given an input frame. 
These inference times are shown in Table \ref{tab_2_time} for each of the three considered techniques, together with the execution time reported in \cite{moccia_learning-based_2018}. 
It should be noticed that the latter was not obtained by running the method in a computer with the same specifications as the first three. 
In particular, inference times in this paper are reported for a GPU-based computation with a NVIDIA GeForce GTX 1060, by testing each model with a batch size of $1$.
Increasing the batch size further decreases the mean inference time due to a better exploitation of the parallel computing capabilities of GPUs.

\begin{table}[h]  %* for the two columns
	%% increase table row spacing, adjust to taste
	\renewcommand{\arraystretch}{1.3}	
	\centering
	%\sisetup{detect-weight=true,detect-inline-weight=math}
\setlength\tabcolsep{3pt}		
\caption{Inference time per frame for the technique from \cite{moccia_learning-based_2018} and the three considered models.}
\begin{tabular}{l cccc}
%\toprule
%& \makecell{Feature-based \\  + SVM \cite{moccia_learning-based_2018}} & Inception V3   & ResNet 50  & SqueezeNet \\
&  SVM \cite{moccia_learning-based_2018} & Inception V3   & ResNet 50  & SqueezeNet \\
\midrule
Time (s) & \ $3.00\mathrm{E}^{-2}$ & $1.70\mathrm{E}^{-2}$ & $8.57\mathrm{E}^{-3}$ & $4.27\mathrm{E}^{-3}$\\
\bottomrule
\end{tabular}
\label{tab_2_time}
\end{table}%* for the two columns









%
%\section{Discussion}
Based on the results obtained in the on-site test phase of this competition we will be able to meaningfully discuss our results.

%\section{Conclusions and Future Work}
Wrapping everything up, summarize what we did here, stress benefits and good performance.

Also, work that could be done now.



\bibliographystyle{abbrv}
%\bibliographystyle{IEEEbib}
%\bibliography{strings,refs}
\bibliography{isbi20_deepdr}

\end{document}